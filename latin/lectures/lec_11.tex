\lecture{10}{2025 February 16 12:23}{2nd Conjugation}
Latin is an inflected language it likes to wrap every word up in a neat 
little package and verb is no exception \\\\
A vern provides us vital clues about the sentence
\begin{enumerate}[I]
  \item Person (who is doing the action)
  \item Numbers (how many us there)
  \item Tense (when we are doing) or voice (active,passive) finally
  \item Mood (indactive or subjunctive)
\end{enumerate}
Just like nouns latin verbs can be catogorized into various types and we call these
types Conjugations \\\\
These classifications based on the theme wovels 
\begin{center}
  -a- -e- -ê- -i- 
\end{center}
and the behaviour of the verb in the present system\\
2nd Conjugation includes all verbs with which add a 
thematic wovel -e- the root to form the present stem \\\\
and for the find present stem we use present infinitive which is a principal parts
and I for the verb conjugation we have to memorize them for example \vspace{5mm} \\
\begin{tabular}{llll}
  \centering
  moneo & monere & monui & monitus \\ 
  Present Active Indicative & Present Infinitive & Perfect Active Indicative & Perfect Passive Participle \\
  I warn & to warn & I warned & having been warned  \\
\end{tabular}
to form the present,imperfect and future tenses we take the second principle part of monere remove the 
-re keep the theme vowel -e- and the personal endings
\begin{center}
 -o,-s,-t,-mus,-tis,-nt 
\end{center}

\begin{center}  
  \captionof{table}{Present}
  \begin{tabular}{lllll}
    \centering
    & Singular & & Plural &  \\
    1st & moneo & I warn & monemus & we warn \\
    2nd & mones & you warn & monetis & you all warn \\ 
    3rd & monet & he/she/it warb & monent & they warh \\
  \end{tabular}
\end{center}

\begin{center}  
  \captionof{table}{Imperfect}
  \begin{tabular}{lllll}
    \centering
    & Singular & & Plural &  \\
    1st & monebam & I was warning & monebamus & we were warning \\
    2nd & monebas & you were warning & monebatis & you all were warning \\ 
    3rd & monebat & he/she/it was warning & monebant & they warning \\
  \end{tabular}
\end{center}
\begin{center}  
  \captionof{table}{Future}
  \begin{tabular}{lllll}
    \centering
    & Singular & & Plural &  \\
    1st & monebo & I will warn & monebimus & we will warn \\
    2nd & monebis & you will warn& monebitis & you all will warn\\ 
    3rd & monebit & he/she/it will warn & monebunt & they will warn\\
  \end{tabular}
\end{center}
to form the perfect,pluperfect and future perfect tenses we take the 
third principle part of monere which is monui remove the -i- and add 
the personal endings
\begin{center}
 -i, -isti, -it, -imus, -istis, -erunt for perfect tense \\
 -eram, -eras, -erat, -eramus, -eratis, -erant for Pluperfect tense \\
 -ero, -eris, -erit, -erimus, -eritis, -erint for Future perfect tense
\end{center}

\begin{center}  
  \captionof{table}{Perfect}
  \begin{tabular}{lllll}
    \centering
    & Singular & & Plural &  \\
    1st & monui & I warned& monuimus & we warned\\
    2nd & monuisti & you warned& monuistis & you all warned\\ 
    3rd & monuit & he/she/it warned & monuerunt & they warned\\
  \end{tabular}
\end{center}

\begin{center}  
  \captionof{table}{Pluperfect}
  \begin{tabular}{lllll}
    \centering
    & Singular & & Plural &  \\
    1st & monueram & I was warned& monueramus & we had warned\\
    2nd & monueras & you had warned& monueratis & you all had warned\\ 
    3rd & monuerat & he/she/it had warned& monuerant & they warned\\
  \end{tabular}
\end{center}


\begin{center}  
  \captionof{table}{Future Perfect}
  \begin{tabular}{lllll}
    \centering
    & Singular & & Plural &  \\
    1st & manuero & I will have warned& monuerimus & we will have warned\\
    2nd & monueris & you will have warned& monueritis & you all will have warned\\ 
    3rd & monuerit & he/she/it will have warned& monuerint & they will have warned\\
  \end{tabular}
\end{center}    
