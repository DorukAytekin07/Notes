\lecture{11}{2025 February 22 13:03}{3rd Declension}
The 3rd Declension is an entirely different beast\\ 
it's the most common declension and contains nouns of all
three genders
\begin{center}
  Femine,Masculine and Neuter 
\end{center}
But it's a whole class of nouns whose nominative forms vary \\ \\ 
So we refer to Nominative as "blank" memorizing each word's 
unique ending. To identify them we look to the genitive for the
-is ending for identification 
\begin{center}
  \large homo,homonis(m.)  
\end{center}
if you see that -is in the genitive then you know you have 3rd
Declension noun. Are you ready? OK so here we go
\begin{center}
  \captionof{table}{Feminine}
  \begin{tabular}{lll}
    & Singular & Plural  \\
    Nom. & - & -es  \\ 
    Genitive & -is & -um \\
    Dative & -i & -ibust \\ 
    Accusative & -em & -es \\ 
    Ablative & -e & -ibus \\
  \end{tabular} 
\end{center}
\begin{center}
  \captionof{table}{Masculine}
  \begin{tabular}{lll}
    & Singular & Plural  \\
    Nom. & - & -es  \\ 
    Genitive & -is & -um \\
    Dative & -i & -ibust \\ 
    Accusative & -em & -es \\ 
    Ablative & -e & -ibus \\
  \end{tabular} 
\end{center}
\begin{center}
  \captionof{table}{Neuter}
  \begin{tabular}{lll}
    & Singular & Plural  \\
    Nom. & - & -a  \\ 
    Genitive & -is & -um \\
    Dative & -i & -ibust \\ 
    Accusative & - & -a \\ 
    Ablative & -e & -ibus \\
  \end{tabular} 
\end{center}
\begin{center}  
  \begin{tabular}{lllll}
    \\
    rex regix (M.) king & & & \\
    & Singular & & Plural &  \\
    Nom. & rex  & the king & reges & the kings \\
    Genitive & regis& of the king & regum & of the kings\\ 
    Dative & regi & to the king & regibus & to the kings \\
    Accusative & regem & the king & reges & the kings \\ 
    Ablative & rege  & by the king & regibus & by the kings\\ 
  \end{tabular}
\end{center}
\begin{center}  
  \begin{tabular}{lllll}
    \\
    uxor uxoris (F.) wife & & & \\
    & Singular & & Plural &  \\
    Nom. & uxor & the wife & uxores & the wifes \\
    Genitive & uxoris & of the wife & uxorum & of the wifes \\ 
    Dative & uxori & to the wife & uxoribus & to the wifes\\
    Accusative & uxorem & the wife & uxores & the wifes\\ 
    Ablative & uxore & by the wife & uxoribus & by the wifes \\ 
  \end{tabular}
\end{center}
\begin{center}  
  \begin{tabular}{lllll}
    \\
    tempus temporis (N.) time & & & \\
    & Singular & & Plural &  \\
    Nom. & tempus & the time & tempora & the times \\
    Genitive & temporis & of the time & temporum & of the times \\ 
    Dative & tempori & to the time & temporibus & to the times\\
    Accusative & tempus & the time & tempora & the times \\ 
    Ablative & tempore & by the time & temporibus & by the times \\ 
  \end{tabular}
\end{center}
\begin{center}
  \huge Deadalus \& Icarus \\ 
  \huge Latin
\end{center}
Deadalus artifex est. Machinas mirabilis facit. Icarus eius filus
est.Fugiti ex Athenis sunt. Insulae Create navigant. Ibi Minonem concurrunt.
Rex est tyrannus. Minos secretum habet. Sua uxor monstrum peperit, Minotaurum.
Minos Daedalum labyrinthum Minotaurum facit. Postea, Daedalus et suus
filius Creatam cum pinnis cereis fugiunt. Daedalus eum manet: noli
volare prope sole. Icarus non verbas patris audit. Pinnae solvunt.
Icarus in marem cadit
\begin{center}
  \huge Deadalus \& Icarus \\ 
  \huge English
\end{center}
Daedalus is an inventor. He makes marvelous machines. Icarus is his son.
Fugitives from Athens, they sail to the island of Crete. There
they encounter Minos. He is a tyrant king. Minos has a secret. His
wife Pasiphae gave birth to the monster, The Minotaur. Minos has
Daedalus build a labyrinth for the Minotaur. Afterwards, Daedalus and Icarus
escape Crete with waxed wings. Daedalus warns him: not to fly too
close to sun. Icarus does not listen to the words of his father. 
The wings melt. Icarus falls into the sea

