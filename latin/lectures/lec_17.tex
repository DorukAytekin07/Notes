\lecture{17}{2025 March 3 10:53}{Part Of Speech}
What is a part of speech the definition of part of speech is
class of words based on their functions they can be list like this
\begin{itemize}
  \item noun 
  \item pronoun
  \item adjective 
  \item conjunction 
  \item verbs 
  \item adverbs 
  \item preposition
  \item interjection
\end{itemize}
\begin{center} 
  \textbf{\title{\huge Nouns}} \\
  A noun is a person,place or thing.
\end{center}
\vspace{5mm}
\begin{center}
  \textbf{\title{\huge Pronoun}} \\ 
  A word used in place of a noun \\
  \begin{tabular}{llll}
    he & is & we & nos \\ 
    she & ea & this one & ille \\ 
    it & id & that one & iste \\ 
  \end{tabular} \\ 
  Caesar crossed the Rubicon \\ 
  He crossed the Rubicon 
\end{center}
\begin{center} 
  \textbf{\title{\huge Adjectives}} \\
  Adjectives modify nouns. Because Latin is an inflected 
  language, your adjectives will have endings that will 
  try to mirror the nouns they modify in gender, number 
  and case. Think of them a bit like groupies, that like
  to dress up to look like their favorite rock band members
\end{center}
\vspace{5mm}
\begin{center}
  \textbf{\title{\huge Conjunctions}} \\ 
  Linking words, that connect other words,phrases or clauses together \\
  \begin{tabular}{lll}
    and & et & ...this and that \\ 
    but & sed & ...stronger but not greater \\ 
    while & dum & he ate while doing his homework
  \end{tabular}
\end{center}
\vspace{5mm}
\begin{center}
  \textbf{\title{\huge Verbs}} \\ 
  Definition:"Verbs are words expressing actions" \\ 
  Examples:typing,eating,running,reading
\end{center}
\vspace{5mm}
\begin{center}
  \textbf{\title{\huge Adverbs}}\\ 
  Adverbs modify verbs, or explain the degree to which 
  someone or something is doing something for example \\ 
  The horse runs swiftly.
\end{center}
\vspace{5mm}
\begin{center}
  \textbf{\title{\huge Prepositons}} \\ 
  Usually a short indeclinable word connecting a noun or 
  noun phrase to another noun, verb or adjactive \\ 
  For example:in sub cum \\ 
  Quintus went to the Forum with Julia \\ 
  Quintus Foro cum Julia venit \\ \vspace{2mm}
  Prepositons usually affect the case of the thing which 
  they refer. In the case of Julia, cum demands Julia take the 
  ablative case -a. We need to study each preposition on a 
  case-by-case basis. Sometimes they shift. In the following 
  example, in silva "in the forest", in demands an ablative 
  but in the example in hostes incedimus, "we march against
  the enemy" in demands an accusative -es 
\end{center}
\vspace{5mm}
\begin{center}
  \textbf{\title{\huge Interjection}} \\ 
  Words that indicate excitement or suprise! \\ 
  Astonishment: o,en,ecce,ehem,vah,Mehercule,pol (Polux)\\ 
  Sorrow:Heu,eheu,vae "Alas" \\ 
  Joy:io,evae,evoe,euhoe \\ 
  Praise:meia,euge 
\end{center}
