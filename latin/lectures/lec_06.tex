\lecture{6}{2025 February 15 10:26}{1st Conjugation}
Latin is an inflected language it likes to wrap every word up in a neat 
little package and verb is no exception \\\\
A vern provides us vital clues about the sentence
\begin{enumerate}[I]
  \item Person (who is doing the action)
  \item Numbers (how many us there)
  \item Tense (when we are doing) or voice (active,passive) finally
  \item Mood (indactive or subjunctive)
\end{enumerate}
Just like nouns latin verbs can be catogorized into various types and we call these
types Conjugations \\\\
These classifications based on the theme wovels 
\begin{center}
  -a- -e- -ê- -i- 
\end{center}
and the behaviour of the verb in the present system\\
1st Conjugation includes all verbs with which add a thematic wovel -a- the root to form
the present stem \\\\
and for the find present stem we use present infinitive which is a principal parts
and I for the verb conjugation we have to memorize them for example \vspace{5mm} \\
\begin{tabular}{llll}
  \centering
  pugno & pugnare & pugnavi & pugnatus &
  Present Active Indicative & Present Infinitive & Perfect Active Indicative & Perfect Passive Participle \\
  I fight & to fight & I fought & having been fought & \\
\end{tabular}
to form the present tense we take the second principle part of pugnare remove the 
-re keep the theme vowel -a- and the personal endings
\begin{center}
 -o,-s,-t,-mus,-tis,-nt 
\end{center}
\begin{center}  
  \captionof{table}{Present}
  \begin{tabular}{lllll}
    \centering
    & Singular & & Plural &  \\
    1st & pugno & I fight & pugnamus & we fight \\
    2nd & pugnas & you fight & pugnatis & you all fight \\ 
    3rd & pugnat & he/she/it fight & pugnant & they fight \\
  \end{tabular}
\end{center}

