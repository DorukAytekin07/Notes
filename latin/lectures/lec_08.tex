
\lecture{8}{2025 February 16 12:47}{3rd Conjugation}
Latin is an inflected language it likes to wrap every word up in a neat 
little package and verb is no exception \\\\
A vern provides us vital clues about the sentence
\begin{enumerate}[I]
  \item Person (who is doing the action)
  \item Numbers (how many us there)
  \item Tense (when we are doing) or voice (active,passive) finally
  \item Mood (indactive or subjunctive)
\end{enumerate}
Just like nouns latin verbs can be catogorized into various types and we call these
types Conjugations \\\\
These classifications based on the theme wovels 
\begin{center}
  -a- -e- -ê- -i- 
\end{center}
Like the 3rd Declension noun the 3rd Conjugation appeard irregular
and weird\\
but that is because we haven't gotten to know it yet\\\\ 
if you remember 2nd Conjugation had a strong -e- \\
In contrast the 3rd Conjugatin has a weak -e- \\\\
That means you will see its bigger brother -i- drop in for the
present tense as your theme vowel but then that weak -e- will return 
for the imperfect and future tenses\\\\
\begin{tabular}{llll}
  \centering
  curro & currere & cucurri & cursus &
  Present Active Indicative & Present Infinitive & Perfect Active Indicative & Perfect Passive Participle \\
  I run & to run & I ran & having been ran & \\
\end{tabular}
to form the present,imperfect and future tenses we take the second principle part of monere remove the 
-re keep the theme vowel -e- and the personal endings
\begin{center}
 -o,-s,-t,-mus,-tis,-nt 
\end{center}

\begin{center}  
  \captionof{table}{Present}
  \begin{tabular}{lllll}
    \centering
    & Singular & & Plural &  \\
    1st & curro & I run& currimus & we run\\
    2nd & curris & you run& curritis& you all run\\ 
    3rd & currit & he/she/it run & currunt& they run\\
  \end{tabular}
\end{center}

\begin{center}  
  \captionof{table}{Imperfect}
  \begin{tabular}{lllll}
    \centering
    & Singular & & Plural &  \\
    1st & curebam & I was running& currebamus & we were running\\
    2nd & currebas & you were running& currebatis & you all were running\\ 
    3rd & currebat & he/she/it was running& currebant & they running\\
  \end{tabular}
\end{center}
\begin{center}  
  \captionof{table}{Future}
  \begin{tabular}{lllll}
    \centering
    & Singular & & Plural &  \\
    1st & currebo & I will run& curremus & we will run\\
    2nd & curres & you will run& curretis & you all will run\\ 
    3rd & curret & he/she/it will run& current & they will run\\
  \end{tabular}
\end{center}
to form the perfect,pluperfect and future perfect tenses we take the 
third principle part of monere which is monui remove the -i- and add 
the personal endings
\begin{center}
 -i, -isti, -it, -imus, -istis, -erunt for perfect tense \\
 -eram, -eras, -erat, -eramus, -eratis, -erant for Pluperfect tense \\
 -ero, -eris, -erit, -erimus, -eritis, -erint for Future perfect tense
\end{center}

\begin{center}  
  \captionof{table}{Perfect}
  \begin{tabular}{lllll}
    \centering
    & Singular & & Plural &  \\
    1st & cucurri & I ran& cucurrimus & we ran\\
    2nd & cucurristi & you ran& cucurristis & you all ran\\ 
    3rd & cucurrit & he/she/it ran& cucurrerunt & they ran\\
  \end{tabular}
\end{center}

\begin{center}  
  \captionof{table}{Pluperfect}
  \begin{tabular}{lllll}
    \centering
    & Singular & & Plural &  \\
    1st & cucurreram & I was run & cucurreramus & we had run \\
    2nd & cucurreras & you had run & cucurreratis & you all had run \\ 
    3rd & cucurrerat & he/she/it had run & cucurrerant & they run \\
  \end{tabular}
\end{center}


\begin{center}  
  \captionof{table}{Future Perfect}
  \begin{tabular}{lllll}
    \centering
    & Singular & & Plural &  \\
    1st & cucurrero & I will have run & cucurrerimus & we will have run\\
    2nd & cucurreris & you will have run & cucurreritis & you all will have run\\ 
    3rd & cucurrerit & he/she/it will have run & cucurrerint & they will have run \\ 
  \end{tabular}
\end{center}

