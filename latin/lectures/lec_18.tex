\lecture{18}{2025 March 6 17:04}{Relative Pronouns}
In latin, Relative Clauses typically introduced by the Pronouns
qui,quae and quod meaning who,which or that
\begin{center}
  We call this group \\ 
  \Large{Relative Pronouns}
\end{center}
Relative pronouns look back to their antecedent in gender 
and number \\
but look forward to their own cluase for case for example \\ \\ 
The man, who was brave, saved the town \\ 
Vir, qui fuit audax, urbem servavit \\ \\ 
The man, for whom there was hope, saved the town \\ 
Vir, cui spem fuit, urbem servavit \\ \\ 
The book, which I read, was good  \\
Liber, quem legi, bonus fuit 
\begin{table}[!htb]
  \begin{center}
    \begin{tabular}{llll}
      Singular & M & F & N \\ 
      Nom & qui & quae & quod \\  
      Genitive & cuius & cuius & cuius \\ 
      Dative & cui & cui & cui \\ 
      Accusative & quem & quam & quod \\ 
      Ablative & quo & qua & quo 
    \end{tabular}
  \end{center}
  \begin{center}
    \begin{tabular}{llll}
      Plural & M & F & N \\ 
      Nom & qui & quae & quae \\  
      Genitive & quorum & quarum & quorum \\ 
      Dative & quibus & quibus & quibus \\ 
      Accusative & quos  & quas & quae \\ 
      Ablative & quibus & quibus & quibus
    \end{tabular}
  \end{center}
\end{table}
