\lecture{10}{2025 February 18 16:41}{2nd Declension}
At first learning latin can often seem intimidating especially vocabulary
that's why we tend to clasify nouns into five major groups and we call these
groups Declensions \\\\ 
the 2nd declension is concerned primarly with masculine and neuter nouns\\\\
nouns ending in -us,-er and -ir are masculine \\
nouns ending in -um are neuter \\\\
the earlier forms of the nominative and accusative were -os and -om
as a result you, will notice that often in inscriptions, terminal
-s and -m are sometimes omitted \\\\ 
Another curiosity of Latin is that many names of countries 
and towns in -us (-os) are feminine for example: Aegyptus, Cyprus \\\\ 
let's take a look at the 2nd Declension 
\begin{center}  
  \captionof{table}{2nd Declension}
  \begin{tabular}{lllll}
    equus -i (M.) horse & & & \\
    & Singular & Plural &  \\
    Nom. & equus & equi  \\
    Genitive & equi & equorum \\ 
    Dative & equo & equis \\
    Accusative & equum & equos \\ 
    Ablative & equo & equis \\ 
  \end{tabular}
\end{center}
Let's look at also a neuter noun
\begin{center}  
  \captionof{table}{2nd Declension}
  \begin{tabular}{lllll}
    bellum  -i (M.) horse & & & \\
    & Singular & Plural &  \\
    Nom. & bellum & bella \\
    Genitive & belli & bellorum \\ 
    Dative & bello & bellis \\
    Accusative & bellum & bella \\ 
    Ablative & bello & bellis \\ 
  \end{tabular}
\end{center}
Lets put what we know into practice:\\\\
\begin{tabular}{lll}
  & English & Latin & 
  Subject (Nom.) & The man is good & Vir est bonus \\  
  Possesion (Genitive) & The horse of the man & Equus viri \\
  Indirect Object (Dative) & He gives a sword to the man & viro gladium dat \\ 
  Direct Object (Accusative) & The Teacher reads a book & magister librum legit  \\ 
  Object of the preposition (Abl.) & He writes with a stylus & stylo scribet \\ 
\end{tabular}\\\\
Now lets us try some sight translation \\ \\ \newpage
\begin{center}
 \huge The Trojan Horse\\
 Latin
\end{center}
Troiani Graecis bellum pugnant. Bellum decem annis mane. Toriani
de bellum fatigant. Graecus Odysseus consilium cogitat Equum lingum
faciunt, donum Troianis,et ipsos intra occultant. Sacerdos Laocoon
eos non accipere monet. Subito, dui serpentes apparent, et 
eumque eos filios procul trahunt. Quando Troiani accepit Graceos
manent.
\begin{center}
 \huge The Trojan Horse\\
 English
\end{center}
The Trojans are fighting a war with the Greeks. The war is
lastin ten years. The Trojans are tired of war. The Greek Odysseus thinks
of a plan. They build a wooden horse as a gift for the trojans and hide inside.
The Trojans priest Laocoon warned them not accept the gift. Suddenly, 
two serpents appeared and dragged him and his sons away. When the
Trojans accepted the gift the Greeks remained
